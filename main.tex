\documentclass{masterthesis}

\usepackage{hyperref} % links
\usepackage{graphicx}
\usepackage{amsmath}
\usepackage{amssymb}
\usepackage{qcircuit}
\usepackage{braket}
\usepackage{float}
\usepackage[toc,page]{appendix}

\begin{document}

\title{Exercises 8.23 and 9.21 - Nielsen and Chuang}

\author{Lorenzo La Corte}

\advisor{}

\examiner{}

\maketitle

\section*{8.23, Amplitude Damping of a Dual-Rail Qubit}

A dual-rail qubit is defined as
\begin{equation}
    \ket{\psi}=a\ket{01}+b\ket{10}
\end{equation}

Applying $\varepsilon_{A D} \otimes \varepsilon_{A D}$ to $\rho=\ket{\psi}\bra{\psi}$ produces a new state $\rho'$, according to the following operator sum representation:
\begin{equation}
    \varepsilon_{A D} \otimes \varepsilon_{A D}(\rho) = \sum_{i} E_{i} \rho E_{i}^{\dagger} = \rho'
\end{equation}

Where the Kraus operators $E_{i}$ are given by the tensor product of the Kraus operators $E_{j}^{s}$, which are the operation elements of the Amplitude Damping acting on a single qubit:
\begin{equation}
    E_{0}^{s} = \begin{pmatrix} 1 & 0 \\ 0 & \sqrt{1-\gamma} \end{pmatrix}, \quad E_{1}^{s} = \begin{pmatrix} 0 & \sqrt{\gamma} \\ 0 & 0 \end{pmatrix}
\end{equation}

Thus, the Kraus operators $E_{i}$ the Amplitude Damping acting on two qubits are given by:
\begin{equation}
    E_{00} = E_{0}^{s} \otimes E_{0}^{s}, \quad , \quad E_{01} = E_{0}^{s} \otimes E_{1}^{s}, \quad E_{10} = E_{1}^{s} \otimes E_{0}^{s}, \quad E_{11} = E_{1}^{s} \otimes E_{1}^{s}
\end{equation}

\begin{multline}\label{eq:op-el-1}
    E_{00} = \ket{00}\bra{00} + \sqrt{1-\gamma}\ket{01}\bra{01} + \sqrt{1-\gamma}\ket{10}\bra{10} + (1-\gamma)\ket{11}\bra{11} = \\
    = \begin{pmatrix} 1 & 0 & 0 & 0 \\ 0 & \sqrt{1-\gamma} & 0 & 0 \\ 0 & 0 & \sqrt{1-\gamma} & 0 \\ 0 & 0 & 0 & 1-\gamma \end{pmatrix}
\end{multline}
\begin{equation}\label{eq:op-el-2}
    E_{01} = \sqrt{\gamma}\ket{00}\bra{01} + \sqrt{\gamma}\sqrt{1-\gamma}\ket{10}\bra{11}
    = \begin{pmatrix} 0 & \sqrt{\gamma} & 0 & 0 \\ 0 & 0 & 0 & 0 \\ 0 & 0 & 0 & \sqrt{\gamma}\sqrt{1-\gamma} \\ 0 & 0 & 0 & 0 \end{pmatrix}
\end{equation}
\begin{equation}\label{eq:op-el-3}
    E_{10} = \sqrt{\gamma}\ket{00}\bra{01} + \sqrt{\gamma}\sqrt{1-\gamma}\ket{01}\bra{11}  
    = \begin{pmatrix} 0 & 0 & \sqrt{\gamma} & 0 \\ 0 & 0 & 0 & \sqrt{\gamma}\sqrt{1-\gamma} \\ 0 & 0 & 0 & 0 \\ 0 & 0 & 0 & 0 \end{pmatrix}
\end{equation}
\begin{equation}\label{eq:op-el-4}
    E_{11} = \gamma \ket{00}\bra{11}
    = \begin{pmatrix} 0 & 0 & 0 & \gamma \\ 0 & 0 & 0 & 0 \\ 0 & 0 & 0 & 0 \\ 0 & 0 & 0 & 0 \end{pmatrix}
\end{equation}

% If we restrict the operator to the span {$\ket{01}$, $\ket{10}$}, we get the operators:
% \begin{equation}
%     E_{0}^{s} = \sqrt{1-\gamma} I
% \end{equation}
% \begin{equation}
%     E_{1}^{s} = \sqrt{\gamma}|00\rangle\langle 01| = \sqrt{\gamma}\begin{pmatrix} 0 & 1 & 0 & 0 \\ 0 & 0 & 0 & 0 \\ 0 & 0 & 0 & 0 \\ 0 & 0 & 0 & 0 \end{pmatrix}
% \end{equation}
% \begin{equation}
%     E_{2} = \sqrt{\gamma}|00\rangle\langle 10| = \sqrt{\gamma}\begin{pmatrix} 0 & 0 & 1 & 0 \\ 0 & 0 & 0 & 0 \\ 0 & 0 & 0 & 0 \\ 0 & 0 & 0 & 0 \end{pmatrix}
% \end{equation}

We have that
\begin{equation}
    E_{00} \rho E_{00}^{\dagger} = (1-\gamma) \rho
\end{equation}
\begin{equation}
    E_{01} \rho E_{01}^{\dagger} = |a|^2 \gamma \ket{00}\bra{00}
\end{equation}
\begin{equation}
    E_{10} \rho E_{10}^{\dagger} = |b|^2 \gamma \ket{00}\bra{00}
\end{equation}
\begin{equation}
    E_{11} \rho E_{11}^{\dagger} = 0
\end{equation}

And so, the application of these operators to the state $\rho$ produces the state $\rho'$, which is given by:
\begin{equation}\label{eq:ad-dual-rail}
    \rho' = \gamma\begin{pmatrix} 1 & 0 & 0 & 0 \\ 0 & 0 & 0 & 0 \\ 0 & 0 & 0 & 0 \\ 0 & 0 & 0 & 0 \end{pmatrix} + (1-\gamma)\begin{pmatrix} 0 & 0 & 0 & 0 \\ 0 & |a|^{2} & a b^{*} & 0 \\ 0 & a^{*} b & |b|^{2} & 0 \\ 0 & 0 & 0 & 0 \end{pmatrix}
\end{equation}

So, with probability $\gamma$, the state is projected to $\ket{00}$, orthogonal to $\ket{\psi}$, while with probability $1-\gamma$, the state is unchanged. 

% This means that the state $\ket{00}$ can be used to detect amplitude damping errors with measurement operators:
% \begin{equation}
%     M_{0} = \ket{00}\bra{00} \quad \text{projector on} \operatorname{span}\{\ket{00}\}
% \end{equation}
% \begin{equation}
%     M_{1} = \ket{01}\bra{01} + \ket{10}\bra{10} + \ket{11}\bra{11} \quad \text{projector on} \operatorname{span}\{\ket{01}, \ket{10}, \ket{11}\}
% \end{equation}

We can get the same result algebraically applying amplitude damping on the first qubit and second qubit, respectively using the third and fourth qubit for the environment, which both start from the state 0.

Since the circuit of the amplitude damping is the following:
\begin{align}
    \Qcircuit @C=1em @R=1.2em {
        \lstick{\rho} & \qw & \ctrl{1} & \targ & \rstick{\rho'} \qw \\
        \lstick{\ket{0}} & \qw & \gate{R_y(\theta)} & \ctrl{-1} & \meter \qw
      }          
\end{align}

So the one for the amplitude damping on dual-rail is:
\begin{align}
    \Qcircuit @C=1em @R=1.2em {
        & \qw & \ctrl{2} & \targ & \qw & \qw & \qw \\
        & \qw & \qw & \qw & \ctrl{2} & \targ & \qw \\
        \lstick{\ket{0}} & \qw & \gate{R_y(\theta)} & \ctrl{-2} & \qw & \qw & \meter \qw \\
        \lstick{\ket{0}} & \qw & \qw & \qw & \gate{R_y(\theta)} & \ctrl{-2} & \meter \qw
      }          
\end{align}

Algebraically, we start from a tensor product state:
\begin{equation}
    \omega_0 = \ket{\psi}\ket{00} = a\ket{01}\ket{00} + b\ket{10}\ket{00} = a\ket{0100} + b\ket{1000}
\end{equation}

And we apply:
\begin{enumerate}
    \item a controlled rotation on the Y axis (control on first qubit, target on the third qubit) 
    \item a CNOT, with control on third qubit and target on the first
\end{enumerate}

So that
\begin{equation}
    \omega_1 = CX_{3,1} CR_{y}(\theta)_{1,3} \omega_0
\end{equation}

The rotation on the Y axis of $\theta$ is defined as:
\begin{equation}
    R_y(\theta) = \begin{pmatrix} \cos\frac{\theta}{2} & -\sin\frac{\theta}{2} \\ \sin\frac{\theta}{2} & \cos\frac{\theta}{2} \end{pmatrix}
\end{equation}

And acts on $\ket{0}$ as following:
\begin{equation}
    R_y(\theta)\ket{0} = \cos\frac{\theta}{2}\ket{0} + \sin\frac{\theta}{2}\ket{1}
\end{equation}

It rotates one term of the two present in the starting system, as it is the only one with the first qubit set to 1, producing
\begin{equation}
    \omega_1 = CX_{3,1} (a\ket{0100} + b\cos\frac{\theta}{2}\ket{1000} + b\sin\frac{\theta}{2}\ket{1010})
\end{equation}

Then the CNOT flips only the third term, producing
\begin{equation}
    \omega_1 = a\ket{0100} + b\cos\frac{\theta}{2}\ket{1000} + b\sin\frac{\theta}{2}\ket{0010}
\end{equation}

This is the state of the system after having applied amplitude damping on the first qubit.

To get $\omega_2$ we apply again:
\begin{enumerate}
    \item a controlled rotation on the Y axis (control on 2nd qubit, target on the 4th qubit) 
    \item a CNOT, with control on 4th qubit and target on the 2nd
\end{enumerate}

Which is:
\begin{equation}
    \omega_2 = CX_{4,2} CR_{y}(\theta)_{2,4} \omega_1
\end{equation}
\begin{equation}
    \omega_2 = CX_{4,2} (a\cos\frac{\theta}{2}\ket{0100} + a\sin\frac{\theta}{2}\ket{0101} + b\cos\frac{\theta}{2}\ket{1000} + b\sin\frac{\theta}{2}\ket{0010})
\end{equation}
\begin{equation}
    \omega_2 = a\cos\frac{\theta}{2}\ket{0100} + a\sin\frac{\theta}{2}\ket{0001} + b\cos\frac{\theta}{2}\ket{1000} + b\sin\frac{\theta}{2}\ket{0010}
\end{equation}

Now, we have a state the we can call $\ket{\phi}$, and our target is to trace out the environment.
\begin{equation}
    \ket{\phi} = \omega_2 = a\cos\frac{\theta}{2}\ket{0100} + a\sin\frac{\theta}{2}\ket{0001} + b\cos\frac{\theta}{2}\ket{1000} + b\sin\frac{\theta}{2}\ket{0010}
\end{equation}

To do that, we pass to the density matrix representation $\ket{\phi}\bra{\phi}$, which has 16 terms, deriving from the tensor product of the 4 terms of $\ket{\phi}$ with their conjugates:
\begin{align}
    \ket{\phi}\bra{\phi} &= (a\cos\frac{\theta}{2}\ket{0100} + a\sin\frac{\theta}{2}\ket{0001} + b\cos\frac{\theta}{2}\ket{1000} + b\sin\frac{\theta}{2}\ket{0010}) \nonumber \\
    &(a^{*}\cos\frac{\theta}{2}\bra{0100} + a^{*}\sin\frac{\theta}{2}\bra{0001} + b^{*}\cos\frac{\theta}{2}\bra{1000} + b^{*}\sin\frac{\theta}{2}\bra{0010})
\end{align}

We can simplify calculations by taking the trace of the environment while multiplying the terms: doing that, we can discard the terms that don't have matching qubits for the environment, as they will be zero when taking the trace.

In particular, we have 4 terms coming from the multiplications of the terms by themselves (where obviously the environment bits are matching), plus 2 more coming from the multiplication of the first and the third terms of $\phi$, which have the same bits from the environment.

So, as a result:
\begin{align}
    Tr_e(\ket{\phi}\bra{\phi}) &= |a|^2 \cos^2 \frac{\theta}{2}\ket{01}\bra{01} + |a|^2 \sin^2 \frac{\theta}{2}\ket{00}\bra{00} + |b|^2 \cos^2 \frac{\theta}{2}\ket{10}\bra{10} \\
    & + |b|^2 \sin^2 \frac{\theta}{2}\ket{00}\bra{00} + a b^{*} \cos^2 \frac{\theta}{2} \ket{01}\bra{10} + a^{*} b \cos^2 \frac{\theta}{2} \ket{10}\bra{01} 
\end{align}

Which can be simplified to 
\begin{align}
    Tr_e(\ket{\phi}\bra{\phi}) &= \sin^2 \frac{\theta}{2}\ket{00}\bra{00} \\ 
    & + |a|^2 \cos^2 \frac{\theta}{2}\ket{01}\bra{01} + a b^{*} \cos^2 \frac{\theta}{2} \ket{01}\bra{10} \\
    & + a^{*} b \cos^2 \frac{\theta}{2} \ket{10}\bra{01} + |b|^2 \cos^2 \frac{\theta}{2}\ket{10}\bra{10}
\end{align}

If we set $\gamma=sin^2\frac{\theta}{2}$, we get the same result as \hyperref[eq:ad-dual-rail]{eq.13}.

So, we proved that the amplitude damping on 2 qubits is described by the operation elements \hyperref[eq:op-el-1]{$E_{00}$}, \hyperref[eq:op-el-2]{$E_{01}$}, \hyperref[eq:op-el-3]{$E_{10}$}, \hyperref[eq:op-el-4]{$E_{11}$}.

\newpage
\section*{9.21, Relationship between Fidelity and Trace Distance}

When comparing pure states and mixed states it is possible to make a stronger statement with respect to the bound between fidelity and trace distance of two mixed states:
\begin{equation}
    1 - F(\rho, \sigma) \leq D(\rho, \sigma)
\end{equation}

As a matter of fact, prove that:
\begin{equation}
    1 - F(\ket{\psi}, \sigma)^2 \leq D(\ket{\psi}, \sigma)
\end{equation}
Where $\ket{\psi}$ is a pure state and $\sigma$ is a mixed state.

We can start from the definition of fidelity between a pure and a mixed state
% todo: add it

Let $E_m$ set of POVM, such that
\begin{equation}
    p_m = Tr(\ket{\psi}\bra{\psi} E_m)
    q_m = Tr(\sigma E_m)
\end{equation}
where $p_m$ and $q_m$ are the probability distributions of respectively $\ket{\psi}$ and $\sigma$.

We can rewrite $p_m$ as
% todo: add equation that brings to $\bra{\psi}Em\ket{psi}$

And write the fidelity as
\begin{equation}
    F(\ket{\psi}, \rho) = \sum_{m} \sqrt{p_m q_m} 
\end{equation}

As a matter of fact:
% todo: add the math to show the statement, using $\bra{psi}E_m\ket{psi} = 1 iff E_m = \ket{psi}\bra{psi}$

Now, consider the trace distance between the probability distributions $p_m$ and $q_m$:
% todo: add 

This measure is upper bounded by the trace distance between the pure state $\ket{\psi}$ and the mixed state $\rho$:
% todo: add 

We can prove the following:
\begin{equation}
    1 - F(\ket{\psi}, \sigma)^2 = D(p_m, q_m) \leq D(\ket{\psi}, \sigma)
\end{equation}

In particular, we can start from ? and use $\bra{psi}E_m\ket{psi} = 1 iff E_m = \ket{psi}\bra{psi}$
% todo: add 

\end{document}

