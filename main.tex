\documentclass{masterthesis}

\usepackage{hyperref} % links
\usepackage{graphicx}
\usepackage{amsmath}
\usepackage{amssymb}
\usepackage{qcircuit}
\usepackage{braket}
\usepackage{float}
\usepackage[toc,page]{appendix}

\begin{document}

\title{Exercises 8.23 and 9.21 - Nielsen and Chuang}

\author{Lorenzo La Corte}

\advisor{}

\examiner{}

\maketitle

\section*{8.23, Amplitude Damping of a Dual-Rail Qubit}

A dual-rail qubit is defined as
\begin{equation}
    \ket{\psi}=a\ket{01}+b\ket{10}
\end{equation}

Applying $\varepsilon_{A D} \otimes \varepsilon_{A D}$ to $\rho=\ket{\psi}\bra{\psi}$ produces a new state $\rho'$, according to the following operator sum representation:
\begin{equation}
    (\varepsilon_{A D} \otimes \varepsilon_{A D})(\rho) = \sum_{i} E_{i} \rho E_{i}^{\dagger} = \rho'
\end{equation}

Where the Kraus operators $E_{i}$ are given by the tensor product of the Kraus operators $E_{j}^{s}$, which are the operation elements of the Amplitude Damping acting on a single qubit:
\begin{equation}
    E_{0}^{s} = \begin{pmatrix} 1 & 0 \\ 0 & \sqrt{1-\gamma} \end{pmatrix}, \quad E_{1}^{s} = \begin{pmatrix} 0 & \sqrt{\gamma} \\ 0 & 0 \end{pmatrix}
\end{equation}

Thus, the Kraus operators $E_{i}$ the Amplitude Damping acting on two qubits are given by:
\begin{equation}
    E_{00} = E_{0}^{s} \otimes E_{0}^{s}, \quad E_{01} = E_{0}^{s} \otimes E_{1}^{s}, \quad E_{10} = E_{1}^{s} \otimes E_{0}^{s}, \quad E_{11} = E_{1}^{s} \otimes E_{1}^{s}
\end{equation}

\begin{multline}\label{eq:op-el-1}
    E_{00} = \ket{00}\bra{00} + \sqrt{1-\gamma}\ket{01}\bra{01} + \sqrt{1-\gamma}\ket{10}\bra{10} + (1-\gamma)\ket{11}\bra{11} = \\
    = \begin{pmatrix} 1 & 0 & 0 & 0 \\ 0 & \sqrt{1-\gamma} & 0 & 0 \\ 0 & 0 & \sqrt{1-\gamma} & 0 \\ 0 & 0 & 0 & 1-\gamma \end{pmatrix}
\end{multline}
\begin{equation}\label{eq:op-el-2}
    E_{01} = \sqrt{\gamma}\ket{00}\bra{01} + \sqrt{\gamma}\sqrt{1-\gamma}\ket{10}\bra{11}
    = \begin{pmatrix} 0 & \sqrt{\gamma} & 0 & 0 \\ 0 & 0 & 0 & 0 \\ 0 & 0 & 0 & \sqrt{\gamma}\sqrt{1-\gamma} \\ 0 & 0 & 0 & 0 \end{pmatrix}
\end{equation}
\begin{equation}\label{eq:op-el-3}
    E_{10} = \sqrt{\gamma}\ket{00}\bra{10} + \sqrt{\gamma}\sqrt{1-\gamma}\ket{01}\bra{11}  
    = \begin{pmatrix} 0 & 0 & \sqrt{\gamma} & 0 \\ 0 & 0 & 0 & \sqrt{\gamma}\sqrt{1-\gamma} \\ 0 & 0 & 0 & 0 \\ 0 & 0 & 0 & 0 \end{pmatrix}
\end{equation}
\begin{equation}\label{eq:op-el-4}
    E_{11} = \gamma \ket{00}\bra{11}
    = \begin{pmatrix} 0 & 0 & 0 & \gamma \\ 0 & 0 & 0 & 0 \\ 0 & 0 & 0 & 0 \\ 0 & 0 & 0 & 0 \end{pmatrix}
\end{equation}

We have that
\begin{equation}
    E_{00} \rho E_{00}^{\dagger} = (1-\gamma) \rho
\end{equation}
\begin{equation}
    E_{01} \rho E_{01}^{\dagger} = |a|^2 \gamma \ket{00}\bra{00}
\end{equation}
\begin{equation}
    E_{10} \rho E_{10}^{\dagger} = |b|^2 \gamma \ket{00}\bra{00}
\end{equation}
\begin{equation}
    E_{11} \rho E_{11}^{\dagger} = 0
\end{equation}

And so, the application of these operators to the state $\rho$ produces the state $\rho'$, which is given by:
\begin{equation}\label{eq:ad-dual-rail}
    \rho' = (1-\gamma)\begin{pmatrix} 0 & 0 & 0 & 0 \\ 0 & |a|^{2} & a b^{*} & 0 \\ 0 & a^{*} b & |b|^{2} & 0 \\ 0 & 0 & 0 & 0 \end{pmatrix} + \gamma\begin{pmatrix} 1 & 0 & 0 & 0 \\ 0 & 0 & 0 & 0 \\ 0 & 0 & 0 & 0 \\ 0 & 0 & 0 & 0 \end{pmatrix}
\end{equation}

Thus, with probability $\gamma$, the state is projected to $\ket{00}$, orthogonal to $\ket{\psi}$, while with probability $1-\gamma$, the state is unchanged. 

Notice that the set of operation elements $\{E_{00}, E_{01}, E_{10}, E_{11}\}$, has the same effect of the operators set
\begin{equation}
    E_{0} = \sqrt{1-\gamma}(\ket{01}\bra{01} + \ket{10}\bra{10}) \quad E_{1} = \sqrt{\gamma}|00\rangle\langle 01| \quad E_{2} = \sqrt{\gamma}|00\rangle\langle 10|
\end{equation}
which, in the span ${\ket{01}, \ket{10}}$, satisfies the completeness relation $\sum_{i} E_{i}^{\dagger} E_{i} = I$.

\paragraph*{Theorem}
The application of amplitude damping applied on a dual rail qubit gives a process which can be described by the operation elements \hyperref[eq:op-el-1]{$E_{00}$}, \hyperref[eq:op-el-2]{$E_{01}$}, \hyperref[eq:op-el-3]{$E_{10}$}, \hyperref[eq:op-el-4]{$E_{11}$}, that is, either nothing happens to the qubit, or the qubit is transformed into the state $\ket{00}$. 

\paragraph*{Proof}
Consider the effect of amplitude damping on the first qubit and second qubit separetely, respectively associated to the third and fourth qubit for the environment, both starting from the state 0.

Since the circuit of the amplitude damping on a single qubit is the following:
\begin{align}
    \Qcircuit @C=1em @R=1.2em {
        \lstick{\rho} & \qw & \ctrl{1} & \targ & \rstick{\rho'} \qw \\
        \lstick{\ket{0}} & \qw & \gate{R_y(\theta)} & \ctrl{-1} & \meter \qw
      }          
\end{align}

So the one for the amplitude damping on a dual-rail qubit is:
\begin{align}
    \Qcircuit @C=1em @R=1.2em {
        \lstick{\rho_A} & \qw & \ctrl{2} & \targ & \qw & \qw & \rstick{\rho_A'} \qw  \\
        \lstick{\rho_B} & \qw & \qw & \qw & \ctrl{2} & \targ & \rstick{\rho_B'} \qw  \\
        \lstick{\ket{0}} & \qw & \gate{R_y(\theta)} & \ctrl{-2} & \qw & \qw & \meter \qw \\
        \lstick{\ket{0}} & \qw & \qw & \qw & \gate{R_y(\theta)} & \ctrl{-2} & \meter \qw
      }          
\end{align}
where $\rho_A$ and $\rho_B$ represent the first and second qubit of the dual-rail qubit, respectively.

Let $\omega_0$ be the starting state of the system, $\omega_1$ the state after the amplitude damping on the first qubit, and $\omega_2$ the state after the amplitude damping on the second qubit.

\begin{equation}
    \omega_0 = \ket{\psi}\ket{00} = a\ket{0100} + b\ket{1000}
\end{equation}

$\omega_1$ is obtained applying, in sequence:
\begin{enumerate}
    \item a controlled rotation on the Y axis: $CR_{y}(\theta)_{1,3}$,
    \item a controlled not $CX_{3,1}$
\end{enumerate}
onto the state $\omega_0$
\begin{equation}
    \omega_1 = CX_{3,1} CR_{y}(\theta)_{1,3} (\omega_0)
\end{equation}

Where the rotation on the Y axis of $\theta$ is defined as:
\begin{equation}
    R_y(\theta) = \begin{pmatrix} \cos\frac{\theta}{2} & -\sin\frac{\theta}{2} \\ \sin\frac{\theta}{2} & \cos\frac{\theta}{2} \end{pmatrix}
\end{equation}
acting on $\ket{0}$ as following:
\begin{equation}
    R_y(\theta)\ket{0} = \cos\frac{\theta}{2}\ket{0} + \sin\frac{\theta}{2}\ket{1}
\end{equation}

Only the second term of $\omega_0$ is rotated, as it is the only one with the first qubit set to 1, producing
\begin{equation}
    \omega_1 = CX_{3,1} (a\ket{0100} + b\cos\frac{\theta}{2}\ket{1000} + b\sin\frac{\theta}{2}\ket{1010})
\end{equation}

Then the CNOT flips only the third term, producing
\begin{equation}
    \omega_1 = a\ket{0100} + b\cos\frac{\theta}{2}\ket{1000} + b\sin\frac{\theta}{2}\ket{0010}
\end{equation}

This is the state of the system after having applied amplitude damping on the first qubit.

$\omega_2$ is obtained applying, in sequence, another pair of gates (controlled rotation and controlled not), on the second and fourth qubit:
\begin{equation}
    \omega_2 = CX_{4,2} CR_{y}(\theta)_{2,4} (\omega_1)
\end{equation}
\begin{equation}
    \omega_2 = CX_{4,2} (a\cos\frac{\theta}{2}\ket{0100} + a\sin\frac{\theta}{2}\ket{0101} + b\cos\frac{\theta}{2}\ket{1000} + b\sin\frac{\theta}{2}\ket{0010})
\end{equation}
\begin{equation}
    \omega_2 = a\cos\frac{\theta}{2}\ket{0100} + a\sin\frac{\theta}{2}\ket{0001} + b\cos\frac{\theta}{2}\ket{1000} + b\sin\frac{\theta}{2}\ket{0010}
\end{equation}

Now, let
\begin{equation}
    \ket{\phi} = \omega_2 = a\cos\frac{\theta}{2}\ket{0100} + a\sin\frac{\theta}{2}\ket{0001} + b\cos\frac{\theta}{2}\ket{1000} + b\sin\frac{\theta}{2}\ket{0010}
\end{equation}

Our target is to trace out the environment from this state.

To do that, we pass to the density matrix representation $\ket{\phi}\bra{\phi}$, which has 16 terms, deriving from the tensor product of the 4 terms of $\ket{\phi}$ with their conjugates:
\begin{align}
    \ket{\phi}\bra{\phi} &= (a\cos\frac{\theta}{2}\ket{0100} + a\sin\frac{\theta}{2}\ket{0001} + b\cos\frac{\theta}{2}\ket{1000} + b\sin\frac{\theta}{2}\ket{0010}) \nonumber \\
    &(a^{*}\cos\frac{\theta}{2}\bra{0100} + a^{*}\sin\frac{\theta}{2}\bra{0001} + b^{*}\cos\frac{\theta}{2}\bra{1000} + b^{*}\sin\frac{\theta}{2}\bra{0010})
\end{align}

We can simplify calculations by taking the trace of the environment while multiplying the terms: we discard the terms that do not have matching qubits for the environment.

In particular, we have 4 terms coming from the multiplications of the terms by themselves (where obviously the environment qubits are matching), plus 2 more coming from the cross multiplication of the first and the third terms of $\ket{\phi}$:

\begin{align}
    Tr_e(\ket{\phi}\bra{\phi}) &= |a|^2 \cos^2 \frac{\theta}{2}\ket{01}\bra{01} + |a|^2 \sin^2 \frac{\theta}{2}\ket{00}\bra{00} + |b|^2 \cos^2 \frac{\theta}{2}\ket{10}\bra{10} \\
    & + |b|^2 \sin^2 \frac{\theta}{2}\ket{00}\bra{00} + a b^{*} \cos^2 \frac{\theta}{2} \ket{01}\bra{10} + a^{*} b \cos^2 \frac{\theta}{2} \ket{10}\bra{01} 
\end{align}

Which can be simplified, using $|a|^2 + |b|^2 = 1$, to 
\begin{align}
    Tr_e(\ket{\phi}\bra{\phi}) &= \sin^2 \frac{\theta}{2}\ket{00}\bra{00} \\ 
    & + |a|^2 \cos^2 \frac{\theta}{2}\ket{01}\bra{01} + a b^{*} \cos^2 \frac{\theta}{2} \ket{01}\bra{10} \\
    & + a^{*} b \cos^2 \frac{\theta}{2} \ket{10}\bra{01} + |b|^2 \cos^2 \frac{\theta}{2}\ket{10}\bra{10}
\end{align}

Let $\gamma=sin^2\frac{\theta}{2}$
\begin{align}
    Tr_e(\ket{\phi}\bra{\phi}) &= \gamma\ket{00}\bra{00} \\ 
    & + (1-\gamma) |a|^2 \ket{01}\bra{01} + (1-\gamma) a b^{*} \ket{01}\bra{10} \\
    & + (1-\gamma) a^{*} b \ket{10}\bra{01} + (1-\gamma) |b|^2 \ket{10}\bra{10} \\
    &= \rho'
\end{align}
thus, we got the same result as \hyperref[eq:ad-dual-rail]{eq:13}, proving that the amplitude damping on 2 qubits is described by the operation elements \hyperref[eq:op-el-1]{$E_{00}$}, \hyperref[eq:op-el-2]{$E_{01}$}, \hyperref[eq:op-el-3]{$E_{10}$}, \hyperref[eq:op-el-4]{$E_{11}$}.

\newpage
\section*{9.21, Relationship between Fidelity and Trace Distance}

The relationship between fidelity and trace distance of two mixed states is defined as
\begin{equation}
    1 - F(\rho, \sigma) \leq D(\rho, \sigma)
\end{equation}

\paragraph*{Theorem}
When comparing pure states and mixed states it is possible to make a stronger statement:
\begin{equation}
    1 - F(\ket{\psi}, \sigma)^2 \leq D(\ket{\psi}, \sigma)
\end{equation}
where $\ket{\psi}$ is a pure state and $\sigma$ is a mixed state.

\paragraph*{Proof}
Consider the definition of fidelity between a pure and a mixed state
\begin{equation}
    F(\ket{\psi}, \sigma) = \sqrt{\bra{\psi}\sigma\ket{\psi}}
\end{equation}

Let $\{E_m\}$ set of POVM, such that we have one and only one term where $E_m = \ket{\psi}\bra{\psi}$, let it be $m=0$, so that $\bra{\psi}E_m\ket{\psi} = 1$. 

We have
\begin{equation}
    p_m = Tr(\ket{\psi}\bra{\psi} E_m) \quad \quad q_m = Tr(\sigma E_m)
\end{equation}
probabilities of obtaining a measurement outcome labeled by $m$.

We can rewrite $p_m$ as
\begin{equation}
    p_m = Tr(\ket{\psi}\bra{\psi} E_m) = \sum_{i} \bra{i} \ket{\psi}\bra{\psi} E_m \ket{i} = \bra{\psi}E_m\ket{\psi}
\end{equation}

And write the fidelity as
\begin{equation}
    F(\ket{\psi}, \rho) = \sum_{m} \sqrt{p_m q_m} 
\end{equation}

As, according to our definition of the POVM, $\bra{\psi}E_m\ket{\psi} = 1$ if and only if $E_m = \ket{\psi}\bra{\psi}$, while it is 0 in the other cases; thus:
\begin{align}
    \sum_{m} \sqrt{p_m q_m} &= \sum_{m} \sqrt{\bra{\psi}E_m\ket{\psi} Tr(\sigma E_m)} = \\
    &= \sqrt{\bra{\psi}E_0\ket{\psi} Tr(\sigma E_0)} = \\
    &= \sqrt{\bra{\psi}\ket{\psi}\bra{\psi}\ket{\psi} Tr(\sigma \ket{\psi}\bra{\psi})} = \\
    &= \sqrt{\bra{\psi} \sigma \ket{\psi}} = F(\ket{\psi}, \sigma)
\end{align}

Now, consider the trace distance between the probability distributions $p_m$ and $q_m$:
\begin{equation}\label{eq:trace-dist}
    D(p_m, q_m) = \frac{1}{2} \sum_{m} |p_m - q_m|
\end{equation}

From Nielsen and Chuang, Theorem 9.1:
\begin{equation}
    D(\ket{\psi}, \sigma) = \max_{\{E_m\}} D(p_m, q_m)
\end{equation}
where the maximization is over all POVMs $\{E_m\}$.

So we have that the trace distance between $p_m$ and $q_m$ is bounded by the trace distance between the correspondent pure state $\ket{\psi}$ and the mixed state $\rho$:
\begin{equation}
    D(p_m, q_m) \leq D(\ket{\psi}, \sigma)
\end{equation}

We can prove the following:
\begin{equation}
    1 - F(\ket{\psi}, \sigma)^2 = D(p_m, q_m) \leq D(\ket{\psi}, \sigma)
\end{equation}

In particular, we can start from the trace distance between $p_m$ and $q_m$ \hyperref[eq:trace-dist]{(eq:47)}
\begin{align}
    D(p_m, q_m) &= \frac{1}{2} \sum_{m} |p_m - q_m| = \\
    &= \frac{1}{2} \sum_{m=0}^{r} |\bra{\psi}E_m\ket{\psi} - Tr(\sigma E_m)|
\end{align}
where r is the number of elements in the POVM set.

We have that:
\begin{itemize}
    \item For $m=0$, $p_m=\bra{\psi}E_0\ket{\psi} = 1$, and $q_m = Tr(\sigma E_0) = \bra{\psi} \sigma \ket{\psi}$,
    \item For $m>0$, $p_m=\bra{\psi}E_m\ket{\psi} = 0$, and $q_m=Tr(\sigma E_m) > 0$.
\end{itemize}

Thus, we can split the sum in two parts, and rewrite the trace distance as
\begin{align}
    D(p_m, q_m) &= \frac{1}{2} \left(|\bra{\psi}E_0\ket{\psi} - Tr(\sigma E_0)| + | \sum_{m=1}^{r} - Tr(\sigma E_m)|\right) = \\
    &= \frac{1}{2} \left(|\bra{\psi}\ket{\psi}\bra{\psi}\ket{\psi} - \bra{\psi} \sigma \ket{\psi} | + | - \sum_{m=1}^{r} Tr(\sigma E_m)|\right) = \\
    &= \frac{1}{2} \left(|1 - F(\ket{\psi}, \sigma)^2 | + | - \sum_{m=1}^{r} Tr(\sigma E_m)|\right) = \\
    &= \frac{1}{2} \left(1 - F(\ket{\psi}, \sigma)^2 + \sum_{m=1}^{r} Tr(\sigma E_m) \right)
\end{align}

We have that 
\begin{equation}
    \sum_{m=0}^{r} Tr(\sigma E_m) = 1
\end{equation}

Thus
\begin{equation}
    \sum_{m=1}^{r} Tr(\sigma E_m) = 1 - Tr(\sigma E_0) = 1 - F(\ket{\psi}, \sigma)^2
\end{equation}

So
\begin{align}
    D(p_m, q_m) &= \frac{1}{2} \left(1 - F(\ket{\psi}, \sigma)^2 + 1 - F(\ket{\psi}, \sigma)^2 \right) = \\
    &= 1 - F(\ket{\psi}, \sigma)^2 
\end{align}

Which proves the theorem:
\begin{equation}
    1 - F(\ket{\psi}, \sigma)^2 = D(p_m, q_m) \leq D(\ket{\psi}, \sigma)
\end{equation}

\end{document}

