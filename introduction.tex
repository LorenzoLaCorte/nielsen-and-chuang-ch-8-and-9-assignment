\chapter{Introduction}
\label{sec:introduction}

This is a template for the Computer Science Master Theses at DIBRIS,
University of Genoa. It uses the \xspace{\LaTeX} class \texttt{masterthesis.cls} by
Davide Ancona. We will use this template to discuss what is generally expected in the
structure of a Master Thesis. While reading academic documents, you will find that
they generally follow a similar structure.

This structure is generally good for research documents, where your main goal is to do
something new. In other cases (e.g., the goal of your thesis is to review a topic and provide
an understanding of the field), a different structure may be more desirable. In any case,
discuss the structure of your thesis with your advisor.

\section{Style and Language}

We are writing a scientific document, and we should consider that skepticism
is at the core of the scientific method. We should be skeptical of our work, and
we can see the thesis as an effort to persuade a skeptical reader that we have done
a piece of work that is both useful and solid. Whenever you state something, you should give proof:
either by referencing one or more authoritative documents proving it or by proving it yourself.
Avoid marketing-speak (``our work is poised to revolutionize...'').\footnote{Well, unless you can
prove what you say. But these statements are generally essentially impossible to prove\ldots}

Don't try to reach a given number of pages. Instead, try to make it
\emph{complete}, \emph{clear} and \emph{concise}. Try to give all the information that will
convince the reader that your work is solid and that you have done a good job; use simple
language and consistent terminology.\footnote{Don't overuse footnotes. By the
way, notice that they should be put after punctuation as in this case.}
If you can say something more simply, do it. Avoid
\href{https://en.wikipedia.org/wiki/Weasel_word}{weasel words} and, when you can, the passive voice.
``We consider this to be not a security risk'' is much clearer than ``This is not considered a
security risk'' because it makes it explicit who makes the claim, and takes the responsibility for
it.

Consider that your reader will be a fellow computer scientist, but not necessarily an expert
of your specific topic. Try to give them the information they need to understand your work, but
do not lose your reader's time and attention with unnecessary details.

Documents like this generally follow a structure like this one. It is not a strict
requirement, but it is a good idea to follow it, as readers will find it easier to navigate it.
Consider that in scientific documents some readers will not read the whole document,
but will look for specific pieces of information. A clear structure will help them.

Chapters generally start with a short introduction, where the reader is introduced to the topic
of the chapter. This is followed by the main content of the chapter, and then by a conclusion
that summarizes the main points of the chapter. Note that both chapter and section titles are
\href{https://www.grammarly.com/blog/capitalization-in-the-titles/}{capitalized}.

Be extremely wary of large language models (e.g., Gemini, ChatGPT or Copilot). You take
responsibility for the content of your thesis, and declare you have written it yourself.
If you want a tool that helps you to write in better English, consider using something like
\href{https://grammarly.com}{Grammarly}; it is also supported by the
\href{https://www.overleaf.com}{Overleaf} online \latex editor.

\section{Motivation}

The motivation is the first important thing that you should write in your Introduction.
You can see it as the effort to convince your reader that the problem you are addressing
\emph{matters}. You should mention that the problem is unsolved, but you will discuss that at
length in the Related Work section (\Cref{sec:related}, in this template). Consider that
you're facing a skeptical reader who is asking ``is this useful?''.

In some cases it will be obvious why the problem matters, but in other cases, this will be
one of the most delicate part of your work. Of course, your advisor will help you with this.

\section{Content of the Thesis}

The other main goal of the Introduction is to give the reader an overview of what they will
find in the document. There is no concept of ``spoiler'' in scientific documents: summarize
what the reader will find in every part of the document. This will help them to navigate it
(and to find what they're interested in, if they are reading the whole document).

\section{\latex Tricks}

If you're using \latex to write your document, you can use some tricks to make your life
easier. This document uses the \verb|\Cref| command to reference sections, figures, and tables conveniently, and \verb|\textcite| and
\verb|\authorcite| to cite paper authors. You can also use \verb|\xspace| to define your macros
and avoid spacing problems. Check the source code to see how they are used.

\texttt{rubber} is a good tool to compile \latex documents. You can run the command
\texttt{rubber -d main} to compile this document to PDF, while running \texttt{pdflatex}
and \texttt{biber} (or \texttt{bibtex}) the right number of times.

The \href{https://www.overleaf.com}{Overleaf} online editor is a good tool to synchronize with
your advisor; it exports a git repository so that you can use it both online and offline.

